\documentclass[11pt]{article}
\usepackage{SIunits}
\usepackage{pinyin}
\usepackage{geometry}
\usepackage{diagbox}
\usepackage{indentfirst}
\usepackage{array}
\usepackage{CJK}
\usepackage{latexsym}
\usepackage{amsmath}
\usepackage{graphicx}
\usepackage{float}
\geometry{left=2.0cm,right=2.0cm,top=2.5cm,bottom=2.5cm}
\renewcommand*{\d}{$\degree$}
\linespread{1.2}

\begin{document}
\pagestyle{plain}

\begin{CJK}{UTF8}{gbsn}

\title{Assignment of Stellar Structure and Evolution  Week 0}
\author{Name: Ping Chen\hspace{0.7cm} No:1501110226}
\maketitle

1. My name is 陈平 \Chen2 \Ping2

2. My undergraduate major is Astronomy. Yes, the lecture material was review of material that I have learned. And I think nothing is new for me in Monday's lecture.

3. My interest increased since I got into department of Astronomy at Peking University and I am going to pursue my career in astronomy.

4. From a broad aspect, I am more interested in observational astronomy than the pure theory stuff. Currently I am very interested in time domain astronomy, like supernova, tidal disruption event(TDE). In fact, I am interested in everything related with astronomy...

5. I can deal with most of my problems in my current research work with python. I started to use python one year ago, and I am familiar with matlab with which I finished my graduate project. C language is one of our undergraduate course but I seldom use it in my work.

6. My Toefl score two years ago can still reflect my English level, good at reading and listening, limited at speaking and fair at writing. I can understand you very well in the first lecture.

7. It's a pity that I can't work with you when I was an undergraduate student. You are the first professor that I tried to contact when I was a second-year undergraduate, but you have too many students at that time. So I am excited to know that you are gonna lead us through this course. Currently I am working with Subo on the ASAS-SN project. 

8. (a) ideal gas is an idealized state of gas which obey the following equation of state: $$PV = nRT$$ where:\\ P is the pressure of the gas;\\ V is the volume of the gas;\\ n is the number of moles; \\ R  is the constant for ideal gas; \\ T is the temperature.\\
From the micro perspective, ideal gas is something that we can neglect its volume of individual molecular and the interaction force, attraction force or repulsive force.

  (b)virial theorem describe the relations between different kinds of energy for a self bound system under stable state (in equilibrium). As a simple example, we deal with the gravitational force bound system, where the virial theorem can be expressed  as $$<T> = -\frac{1}{2}<V>$$ where $<T>$ represent the kinetic energy of the system and $<V>$ means the gravitational  potential energy. \\ From a general sense, if a system is bound under some kind of force which arise from potential energy $V(r) = ar^n$, then the virial theorem gives $2<T> = -n <T_{total}>$ where $<T>$ represent the kinetic energy of the system and $<V>$ means total potential energy.

	(c) As far as I know, black body is also an idealized object in physics, which has nothing to do with its color. Black body can absorb any electromagnetic radiation and doesn't give out any of them. If we define $\alpha$ as the fraction of absorbed energy in the total radiation that radiated onto the object, then $\alpha =1$ for black body.
	
	(d) When we talk about energy transport by convection, we usually refer the energy as heat, which is also known as heat convection, one kind of energy transportation along with heat radiation and heat conduction. Heat convection is achieved by carrying heat energy from one place to another place by the carrier.
	
	(e) Kelvin-Helmhoz timescale is the time needed to radiate away a significant fraction of the gravitational energy at its present luminosity(L), which can be expressed as follow:$$t_{KH} = \frac{GM^2}{RL}$$
	(I didn't pay attention to this term in previous study, so I have to look it up online\\ http://www.astro.sunysb.edu/fwalter/AST101/k-h.html)
	
	(f) HR diagram in essence is a plot of a group of stars to show the relationship between the stars' luminosity and their temperature. There are different ways to illustrate the relationship in practice. We can use the color or spectral type to indicate the temperature, and the absolute magnitude to indicate the luminosity if the stars are at the same distance.
	
	(g) Jeans mass come from jeans instability when people consider gravitational collapse within a gaseous cloud. To put it simple, jeans mass is a 
 critical mass as a function of its density and temperature above which the gravitational attraction will overcome thermal pressure, turbulence and maybe magnetic force to collapse. 
	
	(h) Temperatures of stars at different radius are different and we use the temperature of the stellar photosphere as stellar effective temperature because optical depth is the place where the most of observed radiation come from.
	
	(i) Hydrostatic equilibrium is achieved when the fluid is at rest or each point in the fluid has a constant velocity. There must be some kind of force to constrain and stop the fluid from unstable flow or resist the fluid from collapse. The expression for pressure gradient of star is the result of hydrostatic equilibrium, which is also known as stellar  spherically symmetric stellar hydrostatic equilibrium. $$\frac{dp}{dr} = -\frac{Gm(r)\rho(r)}{r^2}$$

\end{CJK}
\end{document}




\[
\left \{ \begin{array}{l}
    \\

\end{array}\right.
\]

\begin{eqnarray}
\end{eqnarray}

\begin{itemize}
\item
\item
\item
\end{itemize}

\begin{enumerate}
\item
\item
\item
\end{enumerate}

\begin{enumerate}
\item
\item
\item
\item
\end{enumerate}

\begin{description}
   \item[]
   \item[]
   \item[]
 \end{description}

 \begin{description}
   \item[] \hfill \\

   \item[] \hfill \\

   \item[] \hfill \\

 \end{description}



\begin{figure}
\centerline{\includegraphics[width=8cm]{.eps}}
\caption{}
\end{figure}

\begin{figure}[!nbp]
  \begin{minipage}[t]{0.5\linewidth}
    \centering
    \includegraphics[width=2.5in]{.eps}
    \caption{}
    \label{fig:side:a}
  \end{minipage}
  \begin{minipage}[t]{0.5\linewidth}
    \centering
    \includegraphics[width=2.5in]{.eps}
    \caption{}
    \label{fig:side:b}
  \end{minipage}
\end{figure}


  \begin{table}[!hbp]
  \begin{center}
    \begin{tabular}{c|c|c|c|c|c|c|c|c}
         \hline
         \diagbox{}{}&&&&&&&&\\
          \hline
         &&&&&&&&\\
         &&&&&&&&\\
         &&&&&&&&\\
         \hline
 \end{tabular}
    \caption{}
    \end{center}
  \end{table}

    \begin{table}[!hbp]
  \begin{center}
    \begin{tabular}{c|c|c|c|c|c|c|c|c}
         \hline
         &&&&&&&&\\
          \hline
         &&&&&&&&\\
         \hline
         &&&&&&&&\\
         \hline
         &&&&&&&&\\
         \hline
 \end{tabular}
    \caption{}
    \end{center}
  \end{table}

